% !TeX root = report_example.tex
\newcommand*{\MyHeaderPath}{.}% This path definition is also passed to inside the header files.
\newcommand*{\PathToAssets}{../assets}%
\newcommand*{\PathToOutput}{../output}%
\newcommand*{\PathToBibFile}{bibliography.bib}%


%%%%%%%%%%%%%%%%%%%%%%%%%%%%%%%%%%%%%%
%% This file is compiled with XeLaTex.
%%%%%%%%%%%%%%%%%%%%%%%%%%%%%%%%%%%%%%

\documentclass[12pt]{article}
%\documentclass[reqno]{amsart}
%\documentclass[titlepage]{amsart}

%% {{{{
%% This package is already loaded by beamer
% https://tex.stackexchange.com/questions/314344/beamer-presentation-compile-error
\usepackage{graphicx}
%% }}}}

%http://tex.stackexchange.com/questions/36797/how-can-i-make-todonotes-use-all-of-the-margin
\usepackage{fullpage}
%\usepackage{showframe}
% \usepackage[paperwidth=210mm,
%             paperheight=297mm,
%             left=50pt,
%             top=50pt,
%             textwidth=345pt,
%             marginparsep=25pt,
%             marginparwidth=150pt,
%             textheight=692pt,
%             footskip=50pt]
%            {geometry}

%% {{{{
% https://tex.stackexchange.com/questions/9796/how-to-add-todo-notes
\usepackage{xargs} % Use more than one optional parameter in a new commands 
\usepackage[dvipsnames]{xcolor}
% \newcommandx{\todoproposal}[2][1=]{\todo[linecolor=Plum,backgroundcolor=Plum!25,bordercolor=Plum,#1]{#2}}
\newcommandx{\todoproposal}[2][1=]{\todo[disable, linecolor=Plum,backgroundcolor=Plum!25,bordercolor=Plum,#1]{#2}}
% \newcommandx{\tododraft}[2][1=]{\todo[#1]{#2}}
\newcommandx{\tododraft}[2][1=]{\todo[disable, #1]{#2}}
% \newcommandx{\thiswillnotshow}[2][1=]{\todo[disable,#1]{#2}}
%% Math environment in todo note
%
% https://tex.stackexchange.com/questions/298404/todonotes-and-reserveda-nested-itemize-enumerate-environments/298405#298405
\newcommand\todoin[2][]{\todo[inline, caption={2do}, #1]{
\begin{minipage}{\textwidth-4pt}#2\end{minipage}}}
% \newcommand\todoin[2][]{\todo[disable, inline, caption={2do}, #1]{
% \begin{minipage}{\textwidth-4pt}#2\end{minipage}}}
%% }}}}


%% {{{{
%http://tex.stackexchange.com/questions/44858/adding-the-word-appendix-to-table-of-contents-in-latex
\usepackage[titletoc, page]{appendix}
%% }}}}

%% {{{{
	
%I'm using this package to put todo notes into my document. Then, I can
%use it to put a list of the todo notes at the end of the document.
\usepackage[textsize=footnotesize]{todonotes}
% \usepackage[disable=true, colorinlistoftodos,prependcaption,textsize=footnotesize]{todonotes}
%This package has some conficts with amsart. To resolve this, I use
%the following code.
\makeatletter
\providecommand\@dotsep{5}
\def\listtodoname{List of Todos}
\def\listoftodos{\@starttoc{tdo}\listtodoname}
\makeatother
%I got this workaround code from the package's documentation:
%http://get-software.net/macros/latex/contrib/todonotes/todonotes.pdf

% %\newcounter{chapter}
% %\numberwithin{section}{chapter}
% \theoremstyle{mydefinition}
% \newtheorem{exercise}{Exercise}
% \newcommand{\newproblem}[2]{\setcounter{exercise}{#1}\addtocounter{exercise}{
% -1}\begin{exercise}#2\end{exercise}}
% \newcommand{\setcontext}[2]{\setcounter{chapter}{#1}\setcounter{section}{#2}}

% \newtheorem*{remark}{Remark}

%% }}}}


%% {{{{
% Bibliography as numbered section
% https://tex.stackexchange.com/questions/88890/how-to-get-the-references-section-to-be-numbered-as-if-it-were-created-via-sect
\usepackage[numbib]{tocbibind}
%% }}}}


%%%%%%%%%%%%5
%%%% >>>>

%%%%%%%%%%%%5


%%%%%%%%%%%%%%%%%%%%%%%%
%% Section Styling
%%%%%%%%%%%%%%%%%%%%%%%%%

 \usepackage{titlesec}

% %\titleformat*{\subsection}{\newpage \Large \bfseries}
% \titleformat*{\subsubsection}{\large\itshape}

% \usepackage[explicit]{titlesec}

% %Start section with new page
% %http://tex.stackexchange.com/questions/9497/start-new-page-with-each-section
% \newcommand{\sectionbreak}{\clearpage}

% %Underlining ruler for subsections
% %http://tex.stackexchange.com/questions/84061/how-can-i-make-a-bold-horizontal-rule-under-each-section-title
% \titleformat{\section}
%   {\normalfont\LARGE\bfseries}
%   {
%   \thesection
%   }
%   {1em}
%   {#1}
%   [{\titlerule[0.8pt]}]

% \titleformat{\subsection}
%   {\normalfont\Large\bfseries}
%   {\thesubsection}
%   {1em}
%   {#1}

% \titleformat{\subsubsection}
%   {\normalfont\normalsize\itshape}
%   {\thesubsubsection}
%   {1em}
%   {#1}


% Change format of \paragraph{...}
% http://tex.stackexchange.com/questions/3881/formatting-a-paragraph-to-look-like-a-section
% \titleformat{\paragraph}[hang]{\normalfont\normalsize\itshape}{\theparagraph}{1em}{}
% \titlespacing*{\paragraph}{0pt}{3.25ex plus 1ex minus .2ex}{1em}


%%%%%%%%%%%%%%%%%%%%%%%%%%%%%%%%%%%%%%%%%%%%%%
%% Change section styling for HW documents %%
%%%%%%%%%%%%%%%%%%%%%%%%%%%%%%%%%%%%%%%%%%%%%%

%%%
% This first method uses the base LaTeX package:
% http://tex.stackexchange.com/questions/85011/section-and-subsection-heading-style

%\def\@seccntformat#1{\csname the#1\endcsname\quad} %default

%\def\@seccntformat#1{Problem \csname the#1\endcsname\quad} 

% %%% 
% % This next one uses the `titlesec` packages. The references I used are
% % here:
% % http://tex.stackexchange.com/questions/140447/changing-section-heading-style
% % http://tex.stackexchange.com/questions/37189/number-subsections-and-subsubsections-but-not-sections
% \usepackage{bookmark}
% \usepackage{titlesec}
% \titleformat{\section}
%   {\normalfont\bfseries\Large}{Problem \thesection}{1em}{}
% \titleformat{\subsection}
%   {\normalfont\large\itshape}{\thesubsection}{1em}{}
% \titleformat{\subsubsection}
%   {\normalfont\normalsize\itshape}{\thesubsubsection}{1em}{}

% % Add proper labels to PDF bookmarks.
% % http://tex.stackexchange.com/questions/156530/how-to-change-the-pdfbookmark-titles-hyperref
% \makeatletter
% \bookmarksetup{%
%   addtohook={%
%     \ifnum\toclevel@section=\bookmarkget{level}\relax
%       \renewcommand*{\numberline}[1]{Problem #1 }%
%     \fi
%   },
% }



%Option to number subsection with letters
% http://tex.stackexchange.com/questions/74529/sections-indexed-with-numbers-subsections-with-letters
% \renewcommand{\thesection}{\Alph{section}}
% \renewcommand{\thesubsubsection}{\thesubsection.\alph{subsubsection}}
% %%%%%%%%%%%%%%%%%%%%%%%%%%%%%%%%%%%%%%%%%%%%%

%% {{{{
%Links, esp. from table of contents
% http://timmurphy.org/2014/03/11/latex-table-of-contents-with-clickable-links/
\usepackage{hyperref}
\hypersetup{
    colorlinks=true, % make the links colored
    linkcolor=blue, % color TOC links in blue
    urlcolor=red, % color URLs in red
    linktoc=all, % 'all' will create links for everything in the TOC
    %citecolor=gray
    citecolor=blue
    }
%% }}}}

\usepackage{setspace}
%% Deal with warnings

% -- Temporarily filter warning. "remreset package is obsolete". This package is likely innocuous.
% https://tex.stackexchange.com/questions/438543/what-to-do-when-an-actively-maintained-package-requires-an-obsolete-package
\RequirePackage{silence}
\WarningFilter{remreset}{The remreset package}

% -- Todo notes margin warning:
\setlength {\marginparwidth }{2cm}

%https://tex.stackexchange.com/questions/3473/blackboard-bold-variants-for-greek-letters
\usepackage[bbgreekl]{mathbbol}

\usepackage{amsmath, amsfonts, amscd, amssymb, amsthm}
\usepackage{url}
\usepackage{enumerate}
\usepackage{float}
\usepackage{bbm}
\usepackage{color,soul}
\usepackage{thmtools}
\usepackage{threeparttable}
\usepackage[bottom]{footmisc}
\usepackage{rotating}

%% {{{{

%% When Caption is too wide:
% https://tex.stackexchange.com/questions/110393/too-wide-figure-caption/110453
% \captionsetup{width=0.8\textwidth}

% \usepackage[skip=4pt,font=footnotesize, width=\textwidth]{caption}
\usepackage[font=small]{caption}
% \usepackage{caption}
\usepackage{subcaption}
%% }}}}

%% {{{{
%% Bibliography
\usepackage[round,sort,comma,authoryear]{natbib}
% Include in main .tex file: \newcommand*{\MyHeaderPath}{..path_to_header_files..}%
% \bibliographystyle{rusnat}
% \bibliographystyle{plainnat}   % this means that the order of references
%% }}}}


\usepackage[export]{adjustbox}

%% Biblatex
% \usepackage[natbib=true]{biblatex}
% \addbibresource{../../My_Collection.bib}


\usepackage{mathtools}


%% {{{{
% Scalable font and better handling of accents, etc.
% https://tex.stackexchange.com/questions/664/why-should-i-use-usepackaget1fontenc
% https://tex.stackexchange.com/a/10708/41208
% \usepackage[T1]{fontenc}
\usepackage{lmodern}
%% }}}}

%% {{{
% Doesn't work with beamer
% https://tex.stackexchange.com/questions/314344/beamer-presentation-compile-error#comment766741_314344
% Anyway, loading the packages in different order solves the issue: load bm last. I suggest the order inputenc, fontenc, babel if needed, lmodern, mathtools and finally bm
\usepackage{bm}
%% }}}



\usepackage{dcolumn}

%% Force LaTeX image to appear in the section in which it's declared [duplicate]
% https://tex.stackexchange.com/questions/32598/force-latex-image-to-appear-in-the-section-in-which-its-declared
\usepackage[section]{placeins}

%% Used for making landscape (sidewaystable) tables
%
% From here: https://latex.org/forum/viewtopic.php?t=1493
% Creates sideways table environment 
\usepackage{rotating}


%% For use in making tables with multiple panels
%
% See here: https://tex.stackexchange.com/questions/27971/tables-with-multiple-panels-in-latex-r-and-sweave
\usepackage{tabularx}% http://ctan.org/pkg/tabularx
\newcolumntype{Y}{>{\raggedleft\arraybackslash}X}% raggedleft column X

%% For use in raggedright text justification in regular tabular tables
%
% See here: https://tex.stackexchange.com/questions/12703/how-to-create-fixed-width-table-columns-with-text-raggedright-centered-raggedlef
\usepackage{array}
\newcolumntype{L}[1]{>{\raggedright\let\newline\\\arraybackslash\hspace{0pt}}m{#1}}
\newcolumntype{C}[1]{>{\centering\let\newline\\\arraybackslash\hspace{0pt}}m{#1}}
\newcolumntype{R}[1]{>{\raggedleft\let\newline\\\arraybackslash\hspace{0pt}}m{#1}}

% How to add a forced line break inside a table cell
% https://tex.stackexchange.com/questions/2441/how-to-add-a-forced-line-break-inside-a-table-cell
\usepackage{pbox}


%% For use in making landscape tables
% https://tex.stackexchange.com/questions/3930/how-to-rotate-landscape-table-page-in-pdf
% An interesting alternative that also works is https://tex.stackexchange.com/questions/37220/how-to-adjust-or-remove-page-numbers-on-a-landscape-page-within-a-portrait-docum
\usepackage{geometry}
\usepackage{pdflscape}


\usepackage{bibentry}
\makeatletter
%http://tex.stackexchange.com/questions/141446/problem-of-duplicate-identifier-when-using-bibentry-and-hyperref
\renewcommand\bibentry[1]{\nocite{#1}{\frenchspacing
\@nameuse{BR@r@#1\@extra@b@citeb}}}
\makeatother


% %%%% <<<<

% %% Professional Looking Code listings
% %% http://stackoverflow.com/questions/741985/latex-source-code-listing-like-in-professional-books

% %\usepackage{color}
% \usepackage{xcolor}
% \usepackage{listings}

% %\usepackage{caption}
% \DeclareCaptionFont{white}{\color{white}}
% \DeclareCaptionFormat{listing}{\colorbox{gray}{\parbox{\textwidth}{#1#2#3}}}
% \captionsetup[lstlisting]{format=listing,labelfont=white,textfont=white}

% %% Alternative Font for Code Listings
% %% http://tex.stackexchange.com/questions/33685/set-the-font-family-for-lstlisting
% \usepackage{courier}

% \lstset{basicstyle=\footnotesize\ttfamily,breaklines=true}
% \lstset{framextopmargin=50pt,frame=bottomline}

% %% Line Numbers
% \lstset
% { %Formatting for code in appendix
%     %language=Matlab,
%     numbers=left,
%     stepnumber=1,
%     showstringspaces=false,
%     tabsize=1,
%     breaklines=true,
%     breakatwhitespace=false,
% }

% %% Line Numbers starting form an arbitrary number
% %% http://tex.stackexchange.com/questions/61030/how-can-i-start-the-line-numbering-of-my-listing-by-an-arbitrary-number

% % \begin{lstlisting}[firstnumber=100]
% % for i:=maxint to 0 do
% % begin
% % { do nothing }
% % end;
% % \end{lstlisting}

% %%%% >>>>

%%%% <<<<

%% Examples of using underset and underbrace

% $$
% \frac{\dd \pi^2}{\dd K_1} = 
%   \underbrace{\pd{\pi^2}{K_1} }_\text{direct effect}
%   + 
%   \underbrace{\pd{\pi^2}{x_1^*} \pd{x_1^*}{K_1}}_\text{strategic effect}
%   + \underset{\text{due to the envelope theorem}}
%   {\cancel{ \pd{\pi^2}{x_2^*}\pd{x_2^*}{K_1}}}
% $$
%%%% >>>>


%% Aligning numbers by decimal points in table columns
%https://tex.stackexchange.com/questions/2746/aligning-numbers-by-decimal-points-in-table-columns
\usepackage{siunitx}
% From here:https://ctan.org/pkg/siunitx
% Don't intepret comma as decimal marker
\sisetup{
input-decimal-markers = .
}



%%%% <<<<
%% Packages for tables
\usepackage{booktabs}
\usepackage{multirow}
%\usepackage[table,xcdraw]{xcolor}
%I got the following one from here: http://stackoverflow.com/questions/790932/how-to-wrap-text-in-latex-tables
\usepackage{tabulary}
%% Example Table
% \begin{figure}[h!]
% \centering
% \begin{tabulary}{\linewidth}{C|C|C}
% \toprule
% Actions & Inflow at time $t$ & Inflow at time $t + \dd t$     \\ \midrule
% Buy and sell call & $-C$ & $C + \dd C$ \\ \hline
% Short $\alpha$ shares & $\alpha S$ & $-\alpha (S + \dd S)$ \\ \hline
% Buy and sell $\beta$ of option $\hat C$ & $-\beta \hat C$ & $\beta (\hat C + \dd \hat C)$ \\ \hline
% Borrow & $C - \alpha S + \beta \hat C$ & $-(1+r)(C - \alpha S + \beta \hat C)$ \\ \hline
% Net & 0 & $C + \dd C - \alpha (S + \dd S) + \beta (\hat C + \dd \hat C)
%  - (1+r) (C - \alpha S + \beta \hat C)$
% \\ \hline
% \bottomrule
% \end{tabulary} 
% \end{figure}

%itemize in table:
%  http://tex.stackexchange.com/questions/150492/how-to-use-itemize-in-table-environment
\newcommand{\tabitem}{~~\llap{\textbullet}~~}

%useful for spacining:
% http://tex.stackexchange.com/questions/267601/invisible-and-unselectable-text
%%%% >>>>

%%%% <<<<<<
%% Text Box:
% http://tex.stackexchange.com/questions/36524/how-to-put-a-framed-box-around-text-math-enviroment
\usepackage{tcolorbox}
%% Example:

% \begin{tcolorbox}
% blablabla
% \begin{align}
% E &= mc^2 & \text{Formula of the universe}
% \end{align}
% Hoaray
% \end{tcolorbox}

%% http://tex.stackexchange.com/questions/172475/how-can-i-define-a-custom-tcolorbox-environment-with-color-as-a-parameter

% new tcolorbox environment
% #1: tcolorbox options
% #2: color
% #3: box title
\newtcolorbox{textbox}[3][]
{
colframe = #2!25,
colback  = #2!10,
coltitle = #2!20!black,  
title    = #3,
#1,
}
%%%% >>>>>>

%%%% <<<<<<
%%%%%%%%%%%%%%%%%%%%%%%%
%% Cancel out math terms
%http://tex.stackexchange.com/questions/75525/how-to-write-crossed-out-math-in-latex
\usepackage[makeroom]{cancel}
%% Examples:
%%
% \verb|\cancel{5y}|:
% \[ x+\cancel{5y}=0\]
% \verb|\bcancel{5y}|:
% \[ x+\bcancel{5y}=0\]
% \verb|\xcancel{5y}|:
% \[ x+\xcancel{5y}=0\]
% \verb|\cancelto{\infty}{5y}|:
% \[ x+\cancelto{\infty}{5y}=0\]

% The first three commands work in text mode also i.e., \cancel{5y}, \bcancel{5y} and 
% \xcancel{5y} works but \verb|\cancelto{\infty}{5y}| is not. 
%%%% >>>>>>

\usepackage{lscape}
%\begin{landscape}
%...
%\end{landscape}

%I'm using this for the \intertext and \shortintertex commands
%to be used with the align environment from amsmath
\usepackage{mathtools}


%%%%%%%%%%%%%%%%
%Custom commands
%%%%%%%%%%%%%%%%


%%%%%%%%%%%%%%%%%%%%%%%%%%%%%%%%%%%%%%%%%%%%%%%%%%%%%
%%Custom commands for common mathematical expressions



%QED "grave" symbol
\renewcommand{\qed}{\begin{flushright} $\square$ \end{flushright}}
%vector of ones
\newcommand{\1}{\mathbbm{1}}
\newcommand{\E}{\mathbbm{E}}
%differential d
\newcommand{\dd}{\, \mathrm{d}}
%variance and covariance
\newcommand{\Var}{\text{Var}}
\newcommand{\Std}{\text{Std}}
\newcommand{\Cov}{\text{Cov}}
\newcommand{\Corr}{\text{Corr}}
\newcommand{\pd}[2]{\frac{\partial #1}{\partial #2}}
\renewcommand{\t}{\prime}


%% Underline, underbar, ubar, munderbar
% http://tex.stackexchange.com/questions/163280/underbar-changing-the-style-of-font-but-bar-not-why
\usepackage{accents}
\newcommand{\ubar}[1]{\underaccent{\bar}{#1}}

%% Rename hats and tildes
\let\smallhat\hat
\renewcommand{\hat}{\widehat}
\let\smalltilde\tilde
\renewcommand{\tilde}{\widetilde}

% https://tex.stackexchange.com/questions/43335/how-to-write-is-distributed-as-under-a-certain-hypothesis
%% iid over \sim
\makeatletter
\newcommand{\distas}[1]{\mathbin{\overset{#1}{\kern\z@\sim}}}%
\newsavebox{\mybox}\newsavebox{\mysim}
\newcommand{\distras}[1]{%
  \savebox{\mybox}{\hbox{\kern3pt$\scriptstyle#1$\kern3pt}}%
  \savebox{\mysim}{\hbox{$\sim$}}%
  \mathbin{\overset{#1}{\kern\z@\resizebox{\wd\mybox}{\ht\mysim}{$\sim$}}}%
  }

%% Deemphasizing Text {{{
% https://latex.org/forum/viewtopic.php?t=17731
\newcommand{\light}[1]{{\footnotesize \textcolor{gray}{#1}}}
% \light{This is grayed-out}  
% }}}

%%%%%%%%%%%%%%%%%%%%%%%%%%%%%%%
%%Document Specific Commands%%%
%%%%%%%%%%%%%%%%%%%%%%%%%%%%%%%%


%%%%% {{{{
% Theroems

% \newcommand{\bm}[1]{\begin{bmatrix}#1 \end{bmatrix}}
\newtheorem{goal}{Project Goal}

% https://en.wikibooks.org/wiki/LaTeX/Theorems#Theorem_counters
%\newtheorem{proposition}{Proposition}
\newtheorem{proposition}{Proposition}
\newtheorem{mylemma}[proposition]{Lemma}
\newtheorem{mycorollary}[proposition]{Corollary} 
\newtheorem{myassumption}[proposition]{Assumption} 
% OLD: Use proposition counter as `parent` %This comes from https://tex.stackexchange.com/a/381593/41208
% \newtheorem{theorem}{Theorem}[section]

\theoremstyle{definition}
\newtheorem{myexample}{Example}[section]
\theoremstyle{definition}
\newtheorem{mydefinition}[proposition]{Definition}


\newcommand\xqed[1]{%
  \leavevmode\unskip\penalty9999 \hbox{}\nobreak\hfill
  \quad\hbox{#1}}
\newcommand\demo{\xqed{$\triangle$}}

%%%%% }}}}


%% Needed for underbraces under matrix
\newlength{\bracewidth}
\newcommand{\myunderbrace}[2]{\settowidth{\bracewidth}{$#1$}#1\hspace*{-1\bracewidth}\smash{\underbrace{\makebox{\phantom{$#1$}}}_{#2}}}


%% Another method for braces under a matrix. This comes from
% https://tex.stackexchange.com/a/102468/41208
\newcommand\undermat[2]{%
  \makebox[0pt][l]{$\smash{\underbrace{\phantom{%
    \begin{matrix}#2\end{matrix}}}_{\text{$#1$}}}$}#2}



%%%% <<<<
%%%%%%%%%%%%%
%%Custom commands for quickly entering pictures

%Fast Picture
%Arguments {File}{ShortCaption}{LongCaption}{Label}
\newcommand{\fastpic}[4]{
\begin{figure}[H]
\centering
\includegraphics[width=3in]{#1}
\caption[#2]
{#3}
\label{#4}
\end{figure}
}

%A bigger picture
\newcommand{\bfastpic}[4]{
\begin{figure}[H]
\centering
\includegraphics[width=5in]{#1}
\caption[#2]
{#3}
\label{#4}
\end{figure}
}

%fast side-by-side picture
% Arguments--
% #1: filename
% #2: subcaption
% #3: subtag
% #4: filename
% #5: subcaption
% #6: subtag
% #7: supercaption
% #8: supertag
\newcommand{\fastpicTwo}[8]{
\begin{figure}[H]
\centering
\begin{subfigure}[b]{0.45\textwidth}
\includegraphics[width=\textwidth]{#1}
\caption{#2}
\label{#3}
\end{subfigure}
    ~ %add desired spacing between images, e. g. ~, \quad, \qquad, \hfill etc. 
      %(or a blank line to force the subfigure onto a new line)
      \begin{subfigure}[b]{0.45\textwidth}
      \includegraphics[width=\textwidth]{#4}
      \caption{#5}
      \label{#6}
      \end{subfigure}
      \caption{#7}\label{#8}
      \end{figure}

      }





\makeatother
%%%%%%%%%%%%%%%%%%%%%%%%%%%%%%%%%%%%%%%%%%%%%%%%%%%%%%%%%%%%%%%%%%%%%%%%%%
%%%%%%%%%%%%%%%%%%%%%%%%%%%%%%%%%%%%%%%%%%%%%%%%%%%%%%%%%%%%%%%%%%%%%%%%%%










 \newcommand{\STARTONE}{1996-01}
 \newcommand{\ENDONE}{2012-01}
 \newcommand{\STARTTWO}{2012-02}
 \newcommand{\ENDTWO}{2019-12}

%\STARTONE
%\ENDONE
%\STARTTWO
%\ENDTWO

\begin{document}

\pagenumbering{gobble}


\title{
The Puzzle of Filtering Index Options
\\{\color{blue} \large University of Chicago, Winter 2024\footnote{Final project for FINM 329; taught by Jeremy Bejarano.}}
}
% {\color{blue} \large Preliminary. Do not distribute.}
% }

\author{
Viren Desai, Harrison Holt, Ian Hammock 
  % \newline 
  % \hspace{3pt} \indent \hspace{3pt}
  % % I am immensely grateful to...
}
% \maketitle

\begin{titlepage}

% \input{cover_letter.tex}
\maketitle
%http://tex.stackexchange.com/questions/141446/problem-of-duplicate-identifier-when-using-bibentry-and-hyperref
%https://stackoverflow.com/questions/51225448/can-i-make-the-entire-table-1-clickable-as-a-reference-in-latex
% \nobibliography*

% Abstract should have fewer than 150 words

\doublespacing
\begin{abstract}
In this report we summarize our efforts to replicate the data filtration process described in Appendix B of \textit{The Puzzle of Index Option Returns} by \citet{constantinides2013}. These filters shape the underlying distribution of implied volatility (``IV'') and moneyness for a large cross-section of SPX index options (1 million+), and were used to build and price 54 option portfolios. Due to the unavailability of SPX option data from 1985 to 1995, we focus our analysis on replicating the filtration results OptionMetrics data from \STARTONE\  to \ENDONE. We then apply these filters to more recent data from \STARTTWO\  to \ENDTWO. Through a sequence of data visualizations, we show that while the paper's intricately constructed data filters may yield elegant results when applied to one time period, these results do not necessarily port over to other time periods. The implications for option pricing models based on such time-fragile data filters would be an interesting follow-up study. Our detailed analysis and code can be readily found on \href{https://tinyurl.com/3psws69d}{Github}\footnote{ \url{https://tinyurl.com/3psws69d}}.  


\end{abstract}


\end{titlepage}

\doublespacing


\pagenumbering{arabic}
\section{Replicating Table B1}

Appendix B of \citet{constantinides2013} outlines three levels of filters applied to millions of SPX call and put options with an intent to minimize quoting errors in the construction of the paper's 54 option portfolios. In this report we will summarize our implementation, and briefly discuss the challenges and differences encountered in our attempted replication and the subsequent reproduction for a later time period. Our results are summarized in \autoref{table:tableB1}. 

\subsection{Level 1 Filters} 


\paragraph{Methodology}

The Level 1 filters comprised of an ``Identical'' filter, to filter out duplicate options in the OptionMetrics data (measured by identical option type, strike, expiration date, and price), and an ``Identical Except Price'' filter (the ``IEP filter''), which aimed to filter out options identical in all respects except price. In these cases, the paper's authors retained options whose quoted T-bill-based IV was closest to its moneyness neighbors, and deleted the rest.

\paragraph{Commentary on Results}

Two issues arose when applying this set of filters. First, there were quite a few options in the IEP filter that have no reported IV. While this was not explicitly addressed in the paper's description of the Level 1 filters, it created an issue with replication because the T-bill-based IV was required to identify the nearest moneyness neighbors. The missing IV option count here was limited to about 5 in the 1996-2012 dataset, but in the 2012-2019 dataset we observed 355,896 options with no reported IV. For the purpose of this filter, if an IV was not reported, it was not chosen as the option with IV closest to the at the money. Additionally, if an option group's ``at the money'' member could not have their IV calculated by numerical methods (described later in this report), all options in that group would be subsequently dropped via the ``Unable to compute IV'' filter. 

Second, an unexplainable difference occurred upon the application of the Volume = 0 filter. In Table B1 of \citet{constantinides2013}, no options have a Volume = 0 in their dataset. However, we observed over two millions options with a Volume = 0. Unfortunately, no more details were given in the manuscript describing this step. In order to not diverge from their data pool we chose to drop 0 options, this is reflected in \autoref{table:tableB1}. 

Further data is included in \autoref{app:lvl1}. 


\subsection {Level 2 Filters}

\paragraph{Methodology}

The Level 2 filters comprised of: 
\begin{itemize}
  \item \textit{Days to Maturity $<$7 or $>$180}: To filter out options with less than a week to expiration (due to erratic price behavior) or more than 180 days to expiration (due to low liquidity).
  \item \textit{IV $<$5\% or $>$100\%}: To filter out options with extreme IV values.
  \item \textit{Moneyness $<$0.8 or $>$1.2}: To filter out options with extreme moneyness values, due to these options having low liquidity and little value beyond the intrinsic value of the option.
  \item \textit{Implied Interest Rate $<$0}: To filter out options with negative implied interest rates, which are likely due to quoting errors. To calculate the implied interest rate, we compute the put-call parity implied interest rate using each option pair's bid-ask midpoints as the price. The put-call parity implied interest rate is the interest rate that makes the put-call parity equation hold.

\begin{equation}  
  \begin{align}
    C-P &= S-Ke^{rT} \\
    e^{rT} &= \frac{(S-C+P)}{K} \\
    r &= \frac{1}{T} \cdot \log\left(\frac{S-C+P}{K}\right)
  \end{align}
  \label{eq:pcpIntRate}
\end{equation}

  \item \textit{Unable to compute IV}
\end{itemize}


an ``Identical'' filter, to filter out duplicate options in the OptionMetrics data (measured by identical option type, strike, expiration date, and price), and an ``Identical Except Price'' filter (the ``IEP filter''), which aimed to filter out options identical in all respects except price. In these cases, the paper's authors retained options whose quoted T-bill-based IV was closest to its moneyness neighbors, and deleted the rest.

\paragraph{Commentary on Results}

Two issues arose when applying this set of filters. First, there were quite a few options in the IEP filter that have no reported IV. While this was not explicitly addressed in the paper's description of the Level 1 filters, it created an issue with replication because the T-bill-based IV was required to identify the nearest moneyness neighbors. The missing IV option count here was limited to about 5 in the 1996-2012 dataset, but in the 2012-2019 dataset we observed 355,896 options with no reported IV. For the purpose of this filter, if an IV was not reported, it was not chosen as the option with IV closest to the at the money. Additionally, if an option group's ``at the money'' member could not have their IV calculated by numerical methods (described later in this report), all options in that group would be subsequently dropped via the ``Unable to compute IV'' filter. 

Second, an unexplainable difference occurs upon the application of the Volume = 0 filter. In Table B1 of \citet{constantinides2013}, no options have a Volume = 0 in their dataset. However, we observed over two millions options with a Volume = 0. Unfortunately, no more details are given in the manuscript describing this step. In order to not diverge from their data pool we choose to drop 0 options, this is reflected in \autoref{table:tableB1}. 

Further data is included in \autoref{app:lvl2}. 


\subsubsection{\STARTONE\ to \ENDONE }
figure tags: 
\autoref{fig:time1lvl2fig1}
\autoref{fig:time1lvl2fig2}
\autoref{fig:time1lvl2fig3}
\autoref{fig:time1lvl2fig4}
\autoref{fig:time1lvl2fig5}

\subsubsection{\STARTTWO\ to \ENDTWO }
figure tags: 
\autoref{fig:time2lvl2fig1}
\autoref{fig:time2lvl2fig2}
\autoref{fig:time2lvl2fig3}
\autoref{fig:time2lvl2fig4}
\autoref{fig:time2lvl2fig5}


\subsection{Level 3} 

The Level 3 filters described in the paper are not as straightforward as the previous two levels, and are comprised of an implied volatility filter (the ``IV filter'') and a put-call parity implied interest rate filter (the ``put-call parity filter'', or ``PCP filter''). In particular, the intricate construction of these filters, and the lack of specific details regarding filter parameters in \citet{constantinides2013}, make the filtered option counts highly sensitive to changes in the filter parameters. Furthermore, since the IV filter and the PCP filter are applied sequentially, errors in replication get compounded downstream. 

Since the Level 3 filters were more complex, for the 

\subsubsection{IV Filter}

The objective of the IV filter was to reduce the prevalence of apparent butterfly arbitrage. A butterfly arbitrage occurs when there is a discrepancy in the IV structure across difference moneyness levels for options having the same expiration date. The construction of the IV filter involved multiple fitting a quadratic polynomial to the observed log volatilities, the and filtering out options whose implied volatilities 


appendix: \autoref{app:lvl3}


\subsubsection{\STARTONE\ to \ENDONE }
figure tags: 
\autoref{fig:time1lvl3fig1}
\autoref{fig:time1lvl3fig2}
\autoref{fig:time1lvl3fig3}
\autoref{fig:time1lvl3fig4}
\subsubsection{\STARTTWO\ to \ENDTWO }
figure tags: 
\autoref{fig:time2lvl3fig1}
\autoref{fig:time2lvl3fig2}
\autoref{fig:time2lvl3fig3}
\autoref{fig:time2lvl3fig4}











\section{Replicating Table2}
\autoref{tab:t2} describes how many options are found, go missing, or expire in the dataset. An option is found if it reappears the next trading day. An option is missing for if it does not reappear the next trading day. Multiple days missing, counts as multiple options missing. Lastly, if an option is lost and expires this is noted as expired. 

We would like to note an interesting aspect of the \STARTONE\ to \ENDONE\ dataset. Over 80\% of the options expire on a Saturday or a non-trading day. To handle this, we push the expiration day to the nearest Friday, presumably the nearest trading day. However, there are quite a few edge cases which would explain the discrepancy between our analysis and \citet{constantinides2013}. Further investigation is required. A short summary of the day distribution is given in \autoref{table:T2days}. 

%To implement this table we use pandas market calendars to produce an array of NYSE trading days. We then assign an integer to each day and merge this to our dataframe. To determine the time an option is lost, we use a relative distance argument and take differences using the integers. This is much faster than subtracting datetime objects with conditionals on if we're getting a trading day. adf


\begin{table}[ht]

\centering
\captionsetup{font={normalsize,bf}}
\caption{Option Expiration days}


\begin{tabular}{lll}
\toprule
 & 1996-01 to 2012-01 & 2012-02 to 2019-12 \\
\midrule
Total Options & 461890 & 3164202 \\
Trading Days & 10\% & 86\% \\
Saturdays & 87\% & 12\% \\
Other Days & 3\% & 2\% \\
\bottomrule
\end{tabular}

\captionsetup{font={normalsize,bf}}
\caption*{Trading days are determined by the NYSE calendar provided by pandas market days. }
\label{table:T2days}
\end{table}






\section{Conclusion} 

Our findings illustrate that seemingly straight forward instructions for filtering data may lead to divergent results. To reduce these errors, we suggest that journals require that computationally intense manuscripts, such as \citet{constantinides2013}, publish their code base along with their findings. Our data acquisition is briefly described in \autoref{app:data}, and our code base can be found our \href{https://github.com/harrypandas/finm-32900_final_project.git}{Github} \footnote{ \url{https://github.com/harrypandas/finm-32900_final_project.git} }.





\newpage
\bibliographystyle{jpe}
\bibliography{bibliography.bib}  % list here all the bibliographies that you need. 
% \bibliography{\PathToBibFile}
% \printbibliography

\newpage

\thispagestyle{empty}
\begin{landscape}

\begin{table}

\centering
 \captionsetup{font={Large}}
\caption{Table B1 Summary}
\resizebox{1.4\textwidth}{!}{
\hspace*{-4cm}


    \begin{tabular}{*{4}{l} *{11}{r} }
       
        
         \multicolumn{4}{c}{}  & \multicolumn{3}{c}{OptionMetrics: 1996-01 to 2012-01}  &  \multicolumn{1}{c}{} & 
         \multicolumn{3}{c}{OptionMetrics:2012-02 to 2019-12}&  \multicolumn{1}{c}{}  &
          \multicolumn{3}{c}{Total}  \\
         \cline{5-7}
                  
         \cline{9-11}
         \cline{13-15}
         
          &  & & & 
          Deleted &  & Remaining & &
          Deleted &  & Remaining & &
          Deleted &  & Remaining 
          \\

       \hline

	
				Starting & & Calls & &
				 & & 1,704,220 & &
				 & & 7,901,901 & &
				 & & 9,606,121 \\
			
				  & & Puts & &
				 & & 1,706,360 & &
				 & & 7,901,427 & &
				 & & 9,607,787 \\
			
				  & & All & &
				 & & 3,410,580 & &
				 & & 15,803,328 & &
				 & & 19,213,908 \\
			
				Level 1 filters & & Identical & &
				0 & &  & &
				277,102 & &  & &
				277,102 & &  \\
			
				  & & Identical except price & &
				10 & &  & &
				2,557,330 & &  & &
				2,557,340 & &  \\
			
				  & & Bid = 0 & &
				272,078 & &  & &
				1,069,116 & &  & &
				1,341,194 & &  \\
			
				  & & Volume = 0 & &
				0 & &  & &
				0 & &  & &
				0 & &  \\
			
				  & & All & &
				 & & 3,138,492 & &
				 & & 11,899,780 & &
				 & & 15,038,272 \\
			
				Level 2 filters & & Days to expiration <7 or >180 & &
				1,297,729 & &  & &
				3,080,910 & &  & &
				4,378,639 & &  \\
			
				  & & IV <5\% or >100\% & &
				16,432 & &  & &
				63,639 & &  & &
				80,071 & &  \\
			
				  & & K/S <0.8 or >1.2 & &
				550,227 & &  & &
				1,987,486 & &  & &
				2,537,713 & &  \\
			
				  & & Implied interest rate < 0 & &
				592,726 & &  & &
				4,421,368 & &  & &
				5,014,094 & &  \\
			
				  & & Unable to compute IV & &
				38,434 & &  & &
				207,215 & &  & &
				245,649 & &  \\
			
				  & & All & &
				 & & 642,944 & &
				 & & 2,139,162 & &
				 & & 2,782,106 \\
			
				Level 3 filters & & IV filter & &
				0 & &  & &
				0 & &  & &
				0 & &  \\
			
				  & & Put-call parity filter & &
				0 & &  & &
				0 & &  & &
				0 & &  \\
			
				  & & All & &
				 & & 642,944 & &
				 & & 2,139,162 & &
				 & & 2,782,106 \\
			

	        \hline
	    \end{tabular}
	
}
\caption*{
  Number of observations that are removed upon application of Appendix B filters. 
}
\label{table:tableB1}
\end{table}

\vfill
\raisebox{-3.5cm}{\makebox[\linewidth]{\thepage}}
\end{landscape}
\newpage

\thispagestyle{empty}
\begin{landscape}

\begin{table}
    \centering
   \captionsetup{font={Large}}
    \caption{ Table 2 Results}
    
\resizebox{1.4\textwidth}{!}{
\hspace*{-4cm}
  

		\begin{tabular}{*{2}{l} *{15}{r} }
		       
		        
		         \multicolumn{2}{c}{}  & \multicolumn{7}{c}{Calls}  &  \multicolumn{1}{c}{} & 
		         \multicolumn{7}{c}{Puts} \\
		         
		          
		         \cline{3-9}
		         \cline{11-17}
		       
		         
		          \multicolumn{1}{l}{Observations} &  \multicolumn{1}{l}{} &
		          \multicolumn{3}{c}{1996-01 to 2012-01} & 
		          \multicolumn{1}{c}{} &
			\multicolumn{3}{c}{2012-02 to 2019-12} & 
			\multicolumn{1}{c}{} &
		          \multicolumn{3}{c}{1996-01 to 2012-01} & 
		          \multicolumn{1}{c}{} &
			\multicolumn{3}{c}{2012-02 to 2019-12} \\
		        

		       \hline
		       
		       \multicolumn{17}{c}{All trading days} \\ 
		       
		       \hline 

	
		Found &   & 
		586,454 &  & 96\% & 
		 & 
		 3,068,050 &  &95\% & 
		 & 
		 611,764 &  & 96\% & 
		 & 
		 3,210,830& &95\% 
		 \\

		
		Missing &   & 
		0 &  & 0\% & 
		 & 
		 0 &  &0\% & 
		 & 
		 0 &  & 0\% & 
		 & 
		 0& &0\% 
		 \\

		
		Expired &   & 
		25,677 &  & 4\% & 
		 & 
		 161,127 &  &5\% & 
		 & 
		 25,185 &  & 4\% & 
		 & 
		 154,825& &5\% 
		 \\

		
        \hline
        
         \multicolumn{17}{c}{Last trading day of the month} \\

	
		Found &   & 
		39,064 &  & 95\% & 
		 & 
		 725,225 & 98\% & 
		 & 
		 40,737 &  & 96\% & 
		 & 
		 745,805& &98\% 
		 \\

		
		Interpolated &   & 
		1,944 &  & 5\% & 
		 & 
		 18,399 & 2\% & 
		 & 
		 1,867 &  & 4\% & 
		 & 
		 17,476& &2\% 
		 \\

		

	        \hline
	    \end{tabular}
	
  }

  \caption*{Tracking the instances options are found, missing or expired.}
\label{tab:t2}
\end{table}
\vfill
\raisebox{-6.5cm}{\makebox[\linewidth]{\thepage}}
\end{landscape}
\newpage


\begin{appendix}


\section{Data}\label{app:data}

Our option data is queried from OptionMetrics provided by Wharton Research Data Services (WRDS). We limit the query to SECID = 108105, S\&P 500 Index - SPX. We use the three month Tbill as our interest rate, this is from the Federal Reserve Board's H15 report supplied by WRDS. 

In comparison to their data, we have pulled 184 more options than them. It is unclear where the discrepancy lies. We assumed we were off by a day however this will truncate or elongate the dataset by over 300 points. We credit the discrepancy to OptionMetrics updating their data to be more accurate. 

The following links contain the documentation and helpful links for the WRDS database: 
\begin{itemize}
\item \href{https://wrds-www.wharton.upenn.edu/pages/support/manuals-and-overviews/optionmetrics/wrds-overview-optionmetrics/}{Option Metrics Overview} 
\item \href{https://wrds-www.wharton.upenn.edu/data-dictionary/optionm_all/opprcd2023/ }{Option Metric Keys}
\item \href{https://wrds-www.wharton.upenn.edu/pages/get-data/optionmetrics/ivy-db-us/options/option-prices/}{Option Metrics Query} 
\item \href{https://wrds-www.wharton.upenn.edu/data-dictionary/frb_all/rates_daily/}{Federal Reserve Report} 
\end{itemize}



\newpage

\section{Level 1 Filter}\label{app:lvl1}


\begin{table}[ht]

\centering
\captionsetup{font={normalsize,bf}}
\caption{\STARTONE\ to \ENDONE\ Summary of Options with No Volume Nor Open Interest}


\input{\PathToOutput/L1_noVol_noInt_\STARTONE_\ENDONE.tex}
\captionsetup{font={normalsize,bf}}
\caption*{
  Number of observations that remain in the \STARTONE\ to \ENDONE\ data with volume and open interest equal to zero, as well as the overlap. 
}
\label{table:time1lvl1T1}
\end{table}


\begin{table}[ht]

\centering
\captionsetup{font={normalsize,bf}}
\caption{\STARTTWO\ to \ENDONE\ Summary of Options with No Volume Nor Open Interest}


\input{\PathToOutput/L1_noVol_noInt_\STARTTWO_\ENDTWO.tex}
\captionsetup{font={normalsize,bf}}
\caption*{
  Number of observations that remain in the data with volume and open interest equal to zero, as well as the overlap. 
}
\label{table:time2lvl1T1}
\end{table}

\autoref{table:time1lvl1T1} and \autoref{table:time2lvl1T1} illustrate that many of the options in the initial and post filtering dataset have no trading volume nor open interest. Importantly, but perhaps not unexpectedly, nearly half of the options in the \STARTONE\ to \ENDONE\ dataset have no trading volume, and around 20\% of options in this dataset have no open interest. The impact of these options in the portfolios of \citet{constantinides2013}, if they are not filtered out of the final analysis, are unclear to us and merits further investigation. 

\newpage

\section{Level 2 Filter}\label{app:lvl2}
\subsection{\STARTONE\ to \ENDONE }

\subsubsection{Effects of filtering Days to Maturity <7 or >180}
\begin{figure}[H] % You can adjust the placement options (htbp) as needed
  \centering
%\captionsetup{font={normalsize,bf}}
% \caption{Effects of filtering Days to Maturity $<$7 or$ >$180}
  \includegraphics[width=\textwidth]{\PathToOutput/L2_\STARTONE_\ENDONE_fig1.png}%{\PathToOutput/pdf_temp.pdf}
\captionsetup{font={normalsize,bf}}
  \caption{Distribution of time to maturity, measured in years from option initial date to expiration date. The graph on the left shows the distribution prior to applying the initial level 2 filter of excluding days to maturity less than 7 and greater than 180. Right shows distribution post filter.}
\label{fig:time1lvl2fig1}
\end{figure}

\begin{figure}[H] % You can adjust the placement options (htbp) as needed
  \centering
%\captionsetup{font={normalsize,bf}}
% \caption{Effects of filtering Days to Maturity $<$7 or$ >$180}
  \includegraphics[width=\textwidth]{\PathToOutput/L2_\STARTONE_\ENDONE_fig2.png}%{\PathToOutput/pdf_temp.pdf}
\captionsetup{font=normalfont}
  \caption{As noted in the paper, the short maturity options tend to move erratically nearing expiration. In Figure 2, post-filter (red) we see a slight reduction of short-term options with a high implied volatility.}
\label{fig:time1lvl2fig2}
\end{figure}

\subsubsection{Effects of filtering IV <5\% or >100\%}
\begin{figure}[H] % You can adjust the placement options (htbp) as needed
  \centering
\captionsetup{font={normalsize,bf}}
  \includegraphics[width=\textwidth]{\PathToOutput/L2_\STARTONE_\ENDONE_fig3.png}%{\PathToOutput/pdf_temp.pdf}
\captionsetup{font=normalfont}
  \caption{Removing option quotes with implied volatilities lower than 5\% or higher than 100\% eliminates extreme values and reduces the skewness of the implied volatility distribution.}
\label{fig:time1lvl2fig3}
\end{figure}


\subsubsection{Effects of filtering on Moneyness <0.8 or >1.2}
\begin{figure}[H] % You can adjust the placement options (htbp) as needed
  \centering
\captionsetup{font={normalsize,bf}}
  \includegraphics[width=\textwidth]{\PathToOutput/L2_\STARTONE_\ENDONE_fig4.png}%{\PathToOutput/pdf_temp.pdf}
\captionsetup{font=normalfont}
  \caption{Removing option quotes with moneyness lower than 0.8 and higher than 1.2 eliminates extreme values. These extreme values potentially have quotation problems or low values.}
  \label{fig:time1lvl2fig4}
\end{figure}


\subsubsection{Effects of filtering out options where we could not compute IV}
\begin{figure}[H] % You can adjust the placement options (htbp) as needed
  \centering
\captionsetup{font={normalsize,bf}}
  \includegraphics[width=\textwidth]{\PathToOutput/L2_\STARTONE_\ENDONE_fig5.png}%{\PathToOutput/pdf_temp.pdf}
\captionsetup{font=normalfont}
  \caption{Through our analysis we found there are cases where we could not compute implied volatility (IV), as a result, the values were NaN. In figure above, there is a clear trend where the percentage of incomputable IVs increase as time to maturity decreases.}
 \label{fig:time1lvl2fig5}
\end{figure}




\newpage
\subsection{\STARTTWO\ to \ENDTWO }

\subsubsection{Effects of filtering Days to Maturity <7 or >180}
\begin{figure}[H] % You can adjust the placement options (htbp) as needed
  \centering
%\captionsetup{font={normalsize,bf}}
% \caption{Effects of filtering Days to Maturity $<$7 or$ >$180}
  \includegraphics[width=\textwidth]{\PathToOutput/L2_\STARTTWO_\ENDTWO_fig1.png}%{\PathToOutput/pdf_temp.pdf}
\captionsetup{font=normalfont}
  \caption{Distribution of time to maturity, measured in years from option initial date to expiration date. The graph on the left shows the distribution prior to applying the initial level 2 filter of excluding days to maturity less than 7 and greater than 180. Right shows distribution post filter.}
\label{fig:time2lvl2fig1}
\end{figure}

\begin{figure}[H] % You can adjust the placement options (htbp) as needed
  \centering
%\captionsetup{font={normalsize,bf}}
% \caption{Effects of filtering Days to Maturity $<$7 or$ >$180}
  \includegraphics[width=\textwidth]{\PathToOutput/L2_\STARTTWO_\ENDTWO_fig2.png}%{\PathToOutput/pdf_temp.pdf}
\captionsetup{font=normalfont}
  \caption{As noted in the paper, the short maturity options tend to move erratically nearing expiration. In Figure 2, post-filter (red) we see a slight reduction of short-term options with a high implied volatility.}
\label{fig:time2lvl2fig2}
\end{figure}

\subsubsection{Effects of filtering IV <5\% or >100\%}
\begin{figure}[H] % You can adjust the placement options (htbp) as needed
  \centering
\captionsetup{font={normalsize,bf}}
  \includegraphics[width=\textwidth]{\PathToOutput/L2_\STARTTWO_\ENDTWO_fig3.png}%{\PathToOutput/pdf_temp.pdf}
\captionsetup{font=normalfont}
  \caption{Removing option quotes with implied volatilities lower than 5\% or higher than 100\% eliminates extreme values and reduces the skewness of the implied volatility distribution.}
\label{fig:time2lvl2fig3}
\end{figure}


\subsubsection{Effects of filtering on Moneyness <0.8 or >1.2}
\begin{figure}[H] % You can adjust the placement options (htbp) as needed
  \centering
\captionsetup{font={normalsize,bf}}
  \includegraphics[width=\textwidth]{\PathToOutput/L2_\STARTTWO_\ENDTWO_fig4.png}%{\PathToOutput/pdf_temp.pdf}
\captionsetup{font=normalfont}
  \caption{Removing option quotes with moneyness lower than 0.8 and higher than 1.2 eliminates extreme values. These extreme values potentially have quotation problems or low values.}
  \label{fig:time2lvl2fig4}
\end{figure}


\subsubsection{Effects of filtering out options where we could not compute IV}
\begin{figure}[H] % You can adjust the placement options (htbp) as needed
  \centering
\captionsetup{font={normalsize,bf}}
  \includegraphics[width=\textwidth]{\PathToOutput/L2_\STARTTWO_\ENDTWO_fig5.png}%{\PathToOutput/pdf_temp.pdf}
\captionsetup{font=normalfont}
  \caption{Through our analysis we found there are cases where we could not compute implied volatility (IV), as a result, the values were NaN. In figure above, there is a clear trend where the percentage of incomputable IVs increase as time to maturity decreases.}
 \label{fig:time2lvl2fig5}
\end{figure}




\newpage
\section{Level 3 Filter}\label{app:lvl3}
\subsection{\STARTONE\ to \ENDONE }

\begin{figure}[H] % You can adjust the placement options (htbp) as needed
  \centering
  \includegraphics[width=\textwidth]{\PathToOutput/L3_\STARTONE_\ENDONE_fig1_post_L2filter.png}%{\PathToOutput/pdf_temp.pdf}
  \caption{Your caption here}
 \label{fig:time1lvl3fig1}
\end{figure}


\begin{figure}[H] % You can adjust the placement options (htbp) as needed
  \centering
  \includegraphics[width=\textwidth]{\PathToOutput/L3_\STARTONE_\ENDONE_fig2_L2fitted_iv.png}%{\PathToOutput/pdf_temp.pdf}
  \caption{Your caption here}
  \label{fig:time1lvl3fig2}
\end{figure}


\begin{figure}[H] % You can adjust the placement options (htbp) as needed
  \centering
  \includegraphics[width=\textwidth]{\PathToOutput/L3_\STARTONE_\ENDONE_fig3_IV_filter_only.png}%{\PathToOutput/pdf_temp.pdf}
  \caption{Your caption here}
 \label{fig:time1lvl3fig3}
\end{figure}


\begin{figure}[H] % You can adjust the placement options (htbp) as needed
  \centering
  \includegraphics[width=\textwidth]{\PathToOutput/L3_\STARTONE_\ENDONE_fig4_IV_and_PCP.png}%{\PathToOutput/pdf_temp.pdf}
  \caption{Your caption here}
 \label{fig:time1lvl3fig4}
\end{figure}

\newpage
\subsection{\STARTTWO\ to \ENDTWO }

\begin{figure}[H] % You can adjust the placement options (htbp) as needed
  \centering
  \includegraphics[width=\textwidth]{\PathToOutput/L3_\STARTTWO_\ENDTWO_fig1_post_L2filter.png}%{\PathToOutput/pdf_temp.pdf}
  \caption{Your caption here}
  \label{fig:time2lvl3fig1}
\end{figure}


\begin{figure}[H] % You can adjust the placement options (htbp) as needed
  \centering
  \includegraphics[width=\textwidth]{\PathToOutput/L3_\STARTTWO_\ENDTWO_fig2_L2fitted_iv.png}%{\PathToOutput/pdf_temp.pdf}
  \caption{Your caption here}
  \label{fig:time2lvl3fig2}
\end{figure}


\begin{figure}[H] % You can adjust the placement options (htbp) as needed
  \centering
  \includegraphics[width=\textwidth]{\PathToOutput/L3_\STARTTWO_\ENDTWO_fig3_IV_filter_only.png}%{\PathToOutput/pdf_temp.pdf}
  \caption{Your caption here}
  \label{fig:time2lvl3fig3}
\end{figure}


\begin{figure}[H] % You can adjust the placement options (htbp) as needed
  \centering
  \includegraphics[width=\textwidth]{\PathToOutput/L3_\STARTTWO_\ENDTWO_fig4_IV_and_PCP.png}%{\PathToOutput/pdf_temp.pdf}
  \caption{Your caption here}
  \label{fig:time2lvl3fig4}
\end{figure}

\newpage


\end{appendix}

\end{document}
